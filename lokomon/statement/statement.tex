\documentclass{oci}
\usepackage[utf8]{inputenc}
\usepackage{lipsum}

\title{Lokómon GO}
\codename{lokomon}

\begin{document}
\begin{problemDescription}
Hace mucho tiempo nuestros abuelos solían comprar doritos y coleccionar los
tazos de Lokómon que venían de regalo en su interior.
Algunos lokómones populares de esa época eran Nelmanpuff, Pijorgechú
y Olonsonder, de tipo música, hipster y fuego, respectivamente.

\begin{center}
	\includegraphics[scale=0.4]{lokomons.jpg}
\end{center}

En este nuevo siglo la tecnología ha avanzado mucho y ahora nos permite tener
nuevas y entretenidas formas de disfrutar con los lokónomes.
Lokómon GO es un novedoso juego de realidad aumentada basado en esta popular
franquicia de criaturas de bolsillo.
El juego consiste en desplazarse por la ciudad con el objetivo de atrapar y
coleccionar lokómones.
Los lokómones solo aparecen en lugares específicos de la ciudad y usando la
información del GPS, la aplicación se asegura que los jugadores estén
efectivamente en el lugar de aparición antes de que puedan atrapar a los
lokómones.

Dado el éxito del juego, los desarrolladores de Lokomón GO están buscando formas
de optimizar la aparición de lokómones de modo que ciertas restricciones sean
satisfechas.
Por ejemplo, a los desarrolladores les gustaría en algunos casos que los
lokómones que aparecen en cierto sector sean los preferidos de los residentes de
esa localidad y en otros casos que solo aparezcan lokómones odiados por los
vecinos.

Afortunadamente, los creadores del juego disponen de una lista con las
preferencias de lokómones por cada vecino del sector.
Por ejemplo, si enumeramos los lokómones de la figura de más arriba como
Nelmanpuff = 1, Pijorgechú = 2 y Olonsonder = 3, la lista de preferencia $3, 1,
2$ indica que el vecino asociado a esta lista prefiere a Olonsonder en primer
lugar, Nelmanpuff en segundo lugar, y que su lokómon menos preferido es
Pijorgechú.

Este problema consiste en resolver distintas subtareas.
En cada subtarea se indicará un objetivo y tu misión es interpretar las
listas de preferencias de los vecinos para decidir cual es el lokómon más
adecuado para aparecer en el sector.
Antes de especificar las subtareas se detallará el formato de entrada y salida
de los casos de prueba, el cual será uniforme para cada una de las subtareas.
% Si más de un Lokómon es el óptimo, deberás seleccionar aquel cuyo índice sea
% menor.
\end{problemDescription}

\newpage
\begin{inputDescription}
La primera línea de la entrada contiene tres enteros $S$, $N$ y $M$ separados
por un espacio.
El entero $S$ ($1 \le S \le 4$) indica el número de la subtarea que
hay que resolver para este caso.
Los detalles de cada subtarea se describen más adelante en este enunciado.
El entero $N$ ($1\leq N \leq 150$) indica la cantidad de lokómones distintos que
existen.
Cada lokómon es identificado con un número entre 1 y $N$.
Finalmente, $M$ ($1\leq M\leq 10000$) indica el número de personas en el sector. 
Las siguientes $M$ líneas describen las listas de preferencias de cada vecino,
de manera tal que la línea $i$-ésima ($1 \leq i \leq M$) describe la lista del
vecino $i$.
% Cada lista consiste en una permutación de los números del 1 al $N$, indicando cada
% Lokómon está ordenado de mayor a menor preferencia.

Considera la siguiente entrada de ejemplo:

\begin{minipage}[c]{\textwidth}
  \begin{center}
      \color{samplegray}
    \begin{tabular}{|l|} 
      \hline
      \begin{minipage}[t]{0.9\textwidth}
        \vskip 0.3pt
        \color{black}
        \begin{verbatim}
1 4 2
3 1 2 4
1 3 4 2
        \end{verbatim}
      \end{minipage}
      \\
      \hline
    \end{tabular}
  \end{center}
\end{minipage}

La entrada indica que se debe cumplir el objetivo asociado a la subtarea 1, en
un sector donde hay $4$ lokómones distintos y en total hay $2$ vecinos.
El lokómon preferido del vecino 1 es el 3, en el segundo lugar de preferencia
tiene al lokómon 1, en tercer lugar al 2 y en último lugar al 4.
Análogamente el segundo vecino ordena sus preferencias de la forma $1 > 3 > 4 > 2$. 

% En todos los casos las constantes $N, M$ satisfarán las siguientes restricciones: $1 \leq N \leq 150$ y $1 \leq M \leq 10000$.
\end{inputDescription}

\begin{outputDescription}
La salida debe consistir en una sola línea conteniendo un entero correspondiente
a la respuesta de la subtarea correspondiente.
Es decir, si $S$ es 1 debes entregar la respuesta a la subtarea 1, si $S$ es 2
la respuesta a la subtarea 2, \dots

\section*{Subtareas y puntaje}
A continuación se describe la estrategia que hay que utilizar para responder
cada una de las subtareas, junto con ejemplos de entrada y salida para cada una.

\subsection*{Subtarea 1 - El favorito (15 puntos)}
En esta subtarea, debes ignorar el orden de los lokómones y simplemente
mirar cuál es el preferido de cada persona.
Para cada caso, debes responder el lokómon que aparece
más veces como la primera preferencia.
En caso de empate, debes responder aquel que tenga el menor índice. 

\renewenvironment{sampleDescription}{\subsubsection*{Ejemplos de entrada y salida}}{}

\begin{sampleDescription}
\sampleIO{sample1-1}
\sampleIO{sample1-2}
\end{sampleDescription}

\subsection*{Subtarea 2 - El despreciado (15 puntos)}

En esta tarea tampoco interesa el orden de los lokómoness, sino solo el lokómon
más odiado por cada persona.
Para cada entrada, debes responder el Lokómon que aparece más veces al último.
En caso de empate, debes entregar aquel de tenga el menor índice.

\begin{sampleDescription}
\sampleIO{sample2}
\end{sampleDescription}

\subsection*{Subtarea 3 - El ganador justo (30 puntos)}

Observa que la solución para la subtarea 1 puede no ser la más deseable
socialmente.
Considera el siguiente ejemplo:

\begin{minipage}[c]{\textwidth}
  \begin{center}
    \color{samplegray}
    \begin{tabular}{|l|} 
      \hline
      \begin{minipage}[t]{0.9\textwidth}
        \vskip 0.3pt
        \color{black}
        \begin{verbatim}
1 4 3
1 2 3 4
1 2 4 3
2 3 4 1
        \end{verbatim}
      \end{minipage}
      \\
      \hline
    \end{tabular}
  \end{center}
\end{minipage}

En este caso la solución utilizando la estrategia de la subtarea 1 sería el
lokómon 1, pues aparece dos veces como el ganador.
Esto dejaría al tercer vecino muy enojado pues tiene al lokómon 1 como el menos
favorito.
En cambio, escoger el lokómon 2 no molesta tanto a los primeros dos vecinos y
deja muy contento al tercero.

Para elegir al lokómon en esta subtarea, todos los vecinos asignarán un puntaje
a cada uno de los lokómones en su lista de preferencia.
% cada lokómon a cada uno de los lokómones de su lista.
% una de las listas se le asignará un puntaje que indica 
Específicamente, el $k$-ésimo lokómon de la lista obtendrá un puntaje igual a
$N-k$.
Notar que cada vecino asignará un puntaje distinto a cada lokómon dependiendo
de su lista de preferencia.
El puntaje total asociado a un lokómon será la suma de los puntajes que obtuvo
por cada vecino.
A continuación se muestra una tabla con el puntaje asignado por cada vecino
a los distintos lokómones, junto con el puntaje total de cada lokómon.
\begin{center}
\begin{tabular}{c|c|c|c|c}
           & Lokómon 1 & Lokómon 2 & Lokómon 3 & Lokómon 4 \\
  \hline
  vecino 1 & 3         & 2         & 1         & 0         \\
  \hline
  vecino 2 & 3         & 2         & 0         & 1         \\
  \hline
  vecino 3 & 0         & 3         & 2         & 1         \\
  \hline\hline
  total    & 6         & 7         & 3         & 2
\end{tabular}
\end{center}
% Denotaremos como $v_{i,k}$ al puntaje del $k$-ésimo

% Por ejemplo, si la lista del vecino $3$ es $2$ $3$ $4$ $1$, se tendría que $v_{3,1} = 0$, $v_{3,2} = 3$, $v_{3,3} = 2$ y $v_{3,4} = 1$.

% Entrega el Lokómon que maximice la suma de valores por sobre todos los vecinos. Es decir, $j$ tal que el valor $V_{j} := \sum_{i = 1}^{M}v_{i,j}$ sea máximo. En caso de empate, responde aquel de menor índice. En el caso anterior, la suma de valores de $2$ es $7$ en tanto que la suma de valores de $1$ es solo $6$.
En esta subtarea debes encontrar el lokómon que tenga el mayor puntaje total
entre todos los lokómones.

\begin{sampleDescription}
\sampleIO{sample3}
\end{sampleDescription}

\subsubsection*{Subtarea 4 - El detector de mentiras (50 puntos)}
Observa ahora que la solución de la tarea 3 tiene el problema de dar incentivos
a mentir.
Por ejemplo, un vecino podría decir que odia un lokómon que realmente aprecia de
modo que este baje en la evaluación y su primera elección aumente la
valoración relativa. 
A continuación daremos un algoritmo para confundir a los jugadores, de forma que
les sea más complicado mentir.

El algoritmo procede iterativamente eliminando un lokómon en cada iteración
hasta que quede solo uno.
El último lokómon en sobrevivir será elegido como el ganador.
Los pasos específicos del algoritmo se muestran a continuación.

\begin{enumerate}
	\item Si hay un solo lokómon restante escogerlo como el ganador y terminar,
    sino continuar con el paso 2.
	\item Calcular el puntaje total de cada uno de los lokómones restantes de
    la misma forma que en la subtarea 3.
	\item Eliminar al lokómon que tenga el menor puntaje total.
    Si hay más de uno con el mínimo puntaje, eliminar aquel con \textbf{mayor
      índice}.
	\item Volver al paso 1.
\end{enumerate}

Para esta subtarea debes entregar en cada caso el lokómon que resulte ganador de
acuerdo al algoritmo anterior.

\begin{sampleDescription}
\sampleIO{sample4}
\end{sampleDescription}

\end{outputDescription}

\end{document}
