\documentclass{oci}
\usepackage[utf8]{inputenc}
\usepackage{lipsum}
%ESTOY EN WINDOWS ASI QUE NO ME LEE EPS
\usepackage{epstopdf}
%QUITENLE LA LINEA DE ARRIBA SI QUIEREN VIVIR

\title{El malvado profesor Moriarki}
\codename{moriarki}

\begin{document}
\begin{problemDescription}
Araya solía ser una niña muy feliz pero cuando entró a 1ero medio todo cambió.
En un principio logró hacer muchas amigas y en general disfrutaba de casi todas
sus asignaturas, todas a excepción de matemáticas.
Araya está convencida de que todo es culpa de su profesor.
Es más, está segura de que su profesor, el señor Moriarki, es un genio malévolo
que disfruta explicando las cosas de la forma más aburrida posible.

Mientras sus compañeras mantienen buenas notas, las notas de Araya van en
descenso constante.
Su última nota en matemáticas fue la peor de todo el colegio, y a causa de esto
está en peligro de repetir el curso.
Y todo por culpa del profesor Moriarki.
Si Araya no se saca un 7.0 en la próxima prueba tendrá que sufrir nuevamente el
martirio que ha sido 1ero medio.
% Y el único profesor de matemáticas de 1ero medio en todo el colegio es el Señor
% Moriarki. Araya está desesperada.

El profesor Moriarki ya anunció como será la próxima prueba.
Las preguntas consistirán en expresiones aritméticas con números y operaciones
de suma y multiplicación.
Los alumnos sólo deberán evaluar la expresión y responder su resultado.
Por ejemplo, el resultado para la expresión que se muestra a continuación es 23.
Notar que las multiplicaciones deben efectuarse antes que las sumas como es usual.
$$
2 + 2 \times 2 + 2\times 3 \times 2 + 5
$$
Parece muy simple, ¿no?.
Sin embargo, Araya sabe que Moriarki es más astuto que 	esto.
La prueba no puede ser tan fácil.
Debe tramar algo.
Algo malvado.

Llegado el día de la prueba, el profesor les dice a sus alumnos que pueden hacer
la prueba con computador.
Todos están muy extrañados.
¿Cómo puede ser que necesiten computador para una prueba tan fácil?
% Luego menciona que las preguntas les serán entregadas en un archivo de texto, lo
% que tiene sentido.
Finalmente, anuncia que los ejercicios en la prueba serán expresiones con cien
números y tendrán solo 2 segundos para responder cada uno de ellos.

Una sensación de desesperación global se manifiesta en todas las compañeras de
Araya.
Todos se preguntan ``¿es en serio?'', ``no puede hacernos esto'', ``quién
habría pensado que el profesor Moriarki sería tan malvado''.
% y otras frases que no sería pertinente decir en este enunciado.
Por un lado Araya se siente satisfecha, pues ahora tiene evidencia para mostrar a sus
compañeras que Moriarki es evidentemente malvado.
Por otro, comienza a desesperarse pues recuerda que necesita un 7.0 para poder
pasar de curso.
?`Podrías ayudar a Araya a pasar de curso?

% - "Tendrán cinco horas antes de que les entregue la prueba. Úsenlas bien."-
% Anuncia el profesor Moriarki.
% Una luz de esperanza. Mientras algunos combaten la ansiedad de no saber qué hacer esas cinco horas con el computador, algunos ya se rindieron y comenzaron a ver Facebook. Sin embargo, Araya es una niña que logra buscar la respuesta a los problemas más imposibles, mientras no sean fracciones, y tiene la ingeniosa idea de hacer un programa que lea el archivo, y entregue la solución a la expresión en el archivo. ¡Ayuda a Araya a pasar de curso y a vengar a sus compañeros!

\end{problemDescription}

% \newpage
\begin{inputDescription}
  La entrada está compuesta de dos líneas.
  La primera línea contiene un entero $N$ ($1\leq N\leq 100$) que indica la
  cantidad de números que hay en la expresión aritmética.
  La segunda línea contiene $2N-1$ enteros describiendo la expresión.
  Los enteros en las posiciones impares representan los números de la expresión,
  mientras que los enteros en las posiciones pares son las operaciones de la
  expresión.
  Un 0 en una posición par representa una suma ($+$), mientras que un 1
  representa una multiplicación ($\times$). 

  \newpage
  Por ejemplo, para representar la expresión

\vspace{-0.5em}
\begin{minipage}{\textwidth}
  \centering
$2 + 2 \times 2 + 2\times 3 \times 2 + 5$
\end{minipage}
\vspace{-1em}
usamos la siguiente secuencia de enteros
\begin{verbatim}
                              2 0 2 1 2 0 2 1 3 1 2 0 5
\end{verbatim}

  Formalmente, la segunda línea es de la forma $k_1$ $c_1$ $k_2$ $c_2$ $\cdots$
  $c_{n-1}$ $k_n$, donde cada $k_i$ corresponde a un entero positivo menor que
  1000 y cada $c_i$ es igual a 0 o a 1.
  % $i \leq n$, donde 0 se lee como $+$ y 1 se lee como $\times$.
  % Ve los casos de ejemplo para mejor claridad.
  % Se garantiza que en todos los casos el resultado de evaluar la expresión será
  % un número menor o igual que $10^9$.
\end{inputDescription}

\begin{outputDescription}
Tu programa debe imprimir una línea con un único entero, el resultado de evaluar la
expresión aritmética.
\end{outputDescription}

\begin{scoreDescription}
  \score{20} Se probarán varios casos donde todas las operaciones son sumas, es decir,
  $c_i = 0$ para todo $i$.
  Ver ejemplo de entrada 1.
  \score{35} Se probarán varios casos donde la expresión primero contiene solo
  multiplicaciones y luego solo sumas, es decir, existe un $m$ ($0 \leq m \leq
  n$) tal que para cada $i < m$ se tiene que $c_i = 1$ y para cada $i \geq m$
  se tiene que $c_i = 0$.
  Ver ejemplo de entrada 2. 
  \score{45} Se probarán varios casos sin restricciones adicionales.
\end{scoreDescription}

\begin{sampleDescription}
\sampleIO{sample3}
\sampleIO{sample2}
\sampleIO{sample1}
\end{sampleDescription}

\end{document}

%  LocalWords:  Moriarki Araya
