\documentclass{oci}
\usepackage[utf8]{inputenc}
\usepackage{lipsum}
%ESTOY EN WINDOWS ASI QUE NO ME LEE EPS
\usepackage{epstopdf}
%QUITENLE LA LINEA DE ARRIBA SI QUIEREN VIVIR

\title{El Malvado Profesor Moriarki}

\begin{document}
\begin{problemDescription}
Araya era una niña muy feliz, hasta que entró a 1ero medio. Hizo varias amigas, le caía bien a sus profesores, y tuvo buenas notas en todos sus ramos, excepto matemáticas. Araya está convencida de que es culpa de su profesor. Es más, está segura de que el Señor Moriarki es un genio malévolo que encuentra la forma precisa de explicar las cosas de tal manera que ella se aburra lo más posible y se quede dormida en clases.

Mientras que las notas de sus compañeras se mantienen iguales que siempre, las notas de Araya van en descenso constante. Su última nota de matemáticas fue la peor de todo el colegio, y está en peligro de repetir el curso. Y todo por culpa del profesor Moriarki. Si no se saca un 7.0 en la próxima prueba tendrá que sufrir el martirio que ha sido 1ero medio por segunda vez. Y el único profesor de matemáticas de 1ero medio en todo el colegio es el Señor Moriarki. Araya está desesperada.

El profesor ya ha anunciado de qué será la próxima prueba. Las preguntas consistirán en una expresión con números y operaciones de suma y multiplicación, y el alumno sólo debe evaluarla, y responder su resultado. Por ejemplo, la expresión puede verse como
$$
2 + 2 \times 2
$$
y el alumno debe dar como respuesta $6$. Parece muy simple, ¿no?.

Sin embargo, Araya sabe que Moriarki es más astuto que esto. La prueba no puede ser tan fácil. Debe tramar algo. Algo malvado.

Llega el día de la prueba y el profesor les dice a sus alumnos que pueden hacer la prueba con computador. Todos están muy extrañados. ¿Cómo puede ser que necesiten computador para una prueba tan fácil? Luego menciona que las preguntas les serán entregadas en un archivo de texto, lo que tiene sentido. Y finalmente, anuncia que tendrán dos segundos para evaluar una expresión matemática con cien números.

Una sensación de desesperación unánime se manifiesta en todos los compañeros de Araya. Se comenzaban a preguntar ``¿es en serio?'', ``no puede hacernos esto'', ``quién habría pensado que el profesor Moriarki sería tan malvado'', y otras frases que no sería pertinente decir en este enunciado. Araya está satisfecha y furiosa a la vez. Sabía que Moriarki era malvado, pero nadie le creía. Ahora todos sabrán que Araya decía la verdad, pero en este momento no puede hacer nada. La prueba ya comenzó y todos sus compañeros están a su propia suerte.

- "Tendrán cinco horas antes de que les entregue la prueba. Úsenlas bien."- Anuncia el profesor Moriarki. Una luz de esperanza. Mientras algunos combaten la ansiedad de no saber qué hacer esas cinco horas con el computador, algunos ya se rindieron y comenzaron a ver Facebook. Sin embargo, Araya es una niña que logra buscar la respuesta a los problemas más imposibles, mientras no sean fracciones, y tiene la ingeniosa idea de hacer un programa que lea el archivo, y entregue la solución a la expresión en el archivo. ¡Ayuda a Araya a pasar de curso y a vengar a sus compañeros!

\end{problemDescription}

\begin{inputDescription}
La entrada consiste en dos líneas. La primera línea es un número $N$ que indica la cantidad de números en la expresión. La segunda es la expresión misma, que consiste de $2N-1$ números. Los números en las posiciones impares representan los mismos números de la expresión, mientras que los números en las posiciones pares son las operaciones de la expresión. 

Formalmente, la segunda línea es $k_1$ $c_1$ $k_2$ $c_2$ $\cdots$ $c_{n-1}$ $k_n$, donde $k_i$ es un número entero y $c_i$ es 0 o 1, para cada $i \leq n$, donde 0 se lee como $+$ y 1 se lee como $\times$. Ve los casos de ejemplo para mejor claridad.

Se garantiza que en todos los casos, la solución será menor a $10^9$.
\end{inputDescription}

\begin{outputDescription}
Tu programa debe devolver un único número entero, la solución a la expresión del input.
\end{outputDescription}

\begin{scoreDescription}
  \score{10} Se probarán casos donde $1 \leq N \leq 100$, $1\leq k_i \leq 1000$, y $c_i = 0$ para cada $i$.
  \score{10} Se probarán casos donde $1 \leq N \leq 100$, $1\leq k_i \leq 1000$ para cada $i$. Además, $c_i = 1$ para cada $i < \frac{N}{2}$, y $c_i = 0$ para cada $i \geq \frac{N}{2}$. Ver ejemplo 2.
  \score{10} Se probarán casos donde $1 \leq N \leq 100$, $1\leq k_i \leq 1000$, y $c_i \in \{0,1\}$ para cada $i$.
\end{scoreDescription}

\begin{sampleDescription}
\sampleIO{sample1}
\sampleIO{sample2}
\end{sampleDescription}

\end{document}
