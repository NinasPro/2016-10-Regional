\documentclass{oci}
\usepackage[utf8]{inputenc}
\usepackage{lipsum}

\title{Organizador de canciones inteligente}
\codename{}

\begin{document}
\begin{problemDescription}
Pedro es un fanático de la música.
Cada momento de su vida intenta acompañarlo con sus canciones favoritas.
Su celular cuenta con más de 300\,000 pistas y al parecer no piensa parar de
seguir añadiendo canciones.

Con el fin de poder organizar su música en listas de reproducción, Pedro
descargó una aplicación llamada OCI (Organizador de Canciones Inteligente).
Con OCI ahora Pedro puede escoger una nueva lista de reproducción cada día y
escucharla mientras camina desde su casa al colegio.
Una vez que selecciona una lista, las canciones en ella se reproducen en orden
una tras otra hasta que Pedro decide detener la reproducción.
% Además, en cualquier momento Pedro puede decidir detener la reproducción.

A Pedro no le gusta detener una canción mientras esta se está reproduciendo.
Por lo tanto, cada vez que llega al colegio no entra a la sala hasta que la
canción que está escuchando termine.
Debido a esto, Pedro ha estado llegando tarde a sus clases.
% y el inspector ya lo ha amenazado con suspenderlo. 
% Debido a esto, Pedro ha estado llegando tarde a clases ya que
% intencionalmente no entra a la sala hasta que la canción que está
% escuchando termine.

Pedro no puede continuar llegando atrasado, pues el inspector amenazó
con suspenderlo si lo volvía a hacer.
Tampoco puede soportar detener una canción a la mitad, así que necesita tu ayuda
para saber qué hacer.
Naturalmente, Pedro quiere escuchar la mayor cantidad de canciones posible,
pero tiene un tiempo limitado.
Dada una lista de reproducción, y la duración de cada una de sus canciones, tu
tarea es determinar cuál es la última canción de la lista que es posible
reproducir por completo dado el tiempo que Pedro tarda en ir desde su casa al
colegio.

% Dada una lista de reproducción Pedro quiere saber 
% Es por eso que te ha pedido a ti que escribas un
% programa para su aplicación que dado un intervalo de tiempo en
% \texttt{horas:minutos} y las duraciones de cada una las canciones de la
% lista, determines si es posible reproducir la lista de reproducción
% completa, y en el caso de que no se pueda, indicar cual es la ultima
% canción que debe reproducir.
\end{problemDescription}

\begin{inputDescription}
  La entrada está compuesta de varias líneas.
  La primera línea contiene dos enteros $H$ y $M$ correspondientes
  respectivamente a la cantidad de horas y minutos que Pedro tarda en ir desde su
  casa al colegio ($0\leq H\leq 99$, $0\leq M\leq 59$).
  La siguiente línea contiene un entero $N$ ($0\leq N\leq 300\,000$)
  correspondiente a la cantidad de canciones en la lista de reproducción.
  Cada una de las siguientes $N$ líneas describe una canción y cada canción es
  identificada con un número entre 1 y $N$ en este mismo orden.
  Cada una de las líneas describiendo una canción contiene dos enteros $M_i$ y $S_i$
  correspondientes a su duración en minutos y segundos ($0\leq
  M_i\leq 99, 0\leq S_i\leq 59$).
\end{inputDescription}

\begin{outputDescription}
  La salida debe consistir en un único entero indicando el número 
  de la última canción que es posible reproducir completamente.
\end{outputDescription}

\begin{scoreDescription}
  \score{50} Se probarán varios casos donde las canciones durarán una cantidad
  exacta de minutos ($S_i = 0$).
  \score{50} Se probarán varios casos sin restricciones adicionales. 
\end{scoreDescription}

\begin{sampleDescription}
\sampleIO{sample1}
\sampleIO{sample2}
\end{sampleDescription}

\end{document}
