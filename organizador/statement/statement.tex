\documentclass{oci}
\usepackage[utf8]{inputenc}
\usepackage{lipsum}

\title{Organizador de canciones inteligente}

\begin{document}
\begin{problemDescription}
Pedro es melómano total, cada momento silencioso de su vida lo elimina
al acompañarlo con sus temas favoritos. Su celular cuenta con más de
300.000 pistas de música y al parecer no quiere parar de seguir
añadiendo canciones a su celular. Con el fin de organizar su música en
listas de reproducción, Pedro descargó una aplicación para su celular
llamada OCI (Organizador de canciones inteligente), la cual arma listas
de reproducción con un orden preciso y ad-hoc a la situación.

Ultimamente Pedro ha estado llegando tarde a clases ya que él
intencionalmente no entra a la sala hasta que la canción que esta
escuchando termine. Es por eso que te ha pedido a ti que escribas un
programa para su aplicación que dado un intervalo de tiempo en
\texttt{horas:minutos} y las duraciones de cada una las canciones de la
lista, determines si es posible reproducir la lista de reproducción
completa, y en el caso de que no se pueda, indicar cual es la ultima
canción que debe reproducir.
\end{problemDescription}

\begin{inputDescription}
El input consiste en un dos números enteros 0\(\leq\)H\(\leq\)99 y
0\(\leq\)M\(\leq\)59, horas y minutos respectivamente en que demora
desde su casa al colegio. Un número entero 0\(\leq\)N \(\leq\) \(10^3\)
que indica el número de canciones la lista de reproducción. Y finalmente
la duración de cada canción en 0\(\leq\)M,
0\(\leq\)S, minutos y segundos respectivamente.
\end{inputDescription}

\begin{outputDescription}
El output consiste en un número entero que indica cual
es la ultima canción que se alcanza a reproducir, el número esta dado por el
orden en que fueron ingresadas, siendo la primera 1, y la ultima N.
\end{outputDescription}

\begin{scoreDescription}
  \score{10} Restricciones Subtarea1
  \score{10} Restricciones Subtarea2
  \score{10} Restricciones Subtarea3
\end{scoreDescription}

\begin{sampleDescription}
\sampleIO{sample1}
\sampleIO{sample2}
\end{sampleDescription}

\end{document}
