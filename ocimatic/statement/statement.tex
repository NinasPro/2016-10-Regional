\documentclass{oci}
\usepackage[utf8]{inputenc}
\usepackage{lipsum}
\usepackage{tabu}
\usepackage{xcolor}

\title{Ocimatic}

\begin{document}
\begin{problemDescription}
La Organización de Comercio Internacional (OCI) desarrolla un software llamado
\emph{ocimatic} capaz de realizar algunas tareas de contabilidad.
Los usuarios de ocimatic pueden interactuar con el software usando lenguaje
natural.
Una típica conversación con ocimatic es de la siguiente forma.

\begin{center}
  Ejemplo de un chat con ocimatic.
% \fbox{\includegraphics[scale=0.45]{ocimatic2.png}}
\end{center}

Ocimatic es un software extremadamente sofisticado e internamente utiliza
complejos algoritmos con el fin de procesar el texto ingresado por los usuarios.
Un algoritmo importante dentro de ocimatic está encargado del reconocimiento de
\emph{entidades nombradas}.
Para ocimatic, una entidad nombrada es cualquier fragmento de un texto que
corresponda al nombre de una persona o al de una organización.
Por ejemplo, en el texto ``\texttt{Gimuel Caspumano es el gerente general
  de la OCI.}''
se pueden reconocer dos entidades nombradas: \emph{Gimuel Caspumano} y \emph{OCI}.

Dado un texto, una entidad nombrada puede representarse con un par de números
$(a,b)$ que indican respectivamente la posición inicial y final en que la
entidad aparece en el texto.
Por ejemplo, dado el texto de ejemplo anterior la entidad \emph{Gimuel Caspumano}
puede representarse con el par (1,16) y la entidad \emph{OCI} con el par
(46,48).
Esto puede apreciarse en la siguiente figura.

\vspace{1em}
% \newcolumntype{C}[1]{>{\centering}m{#1}}
% \newcolumntype{C}{>{\centering\let\newline\\\arraybackslash\hspace{0pt}}p{0.67em}}
\newcolumntype{C}{@{}>{\centering}p{1em}@{}}
% \hspace*{-2.5em}
\hspace*{-4em}
\begin{tabu}{C|C|C|C|C|C|C|C|C|C|C|C|C|C|C|C|C|C|C|C|C|C|C|C|C|C|C|C|C|C|C|C|C|C|C|C|C|C|C|C|C|C|C|C|C|C|C|C|C}
  \taburulecolor{lightgray}
  % \hline
  \rowfont{\scriptsize}
  1&2&3&4&5&6&7&8&9&10&11&12&13&14&15&16&17&18&19&20&21&22&23&24&25&26&27&28&29&30&31&32&33&34&35&36&37&38&39&40&41&42&43&44&45&46&47&48&49\\
  \hline
  \rowfont{\small}
  G&i&m&u&e&l& &C&a&s&p&u&m&a&n&o& &e&s& &e&l& &g&e&r&e&n&t&e& &g&e&n&e&r&a&l& &d&e& &l&a& &O&C&I&.\\
  % \hline
\end{tabu}
\vspace{0.5em}
% El proceso de reconocer entidades es complejo e imperfecto, es por esto que a
% cada entidad reconocida además se la asigna una \emph{confianza}.
% Esta confianza corresponde a un número entre 0 y 100 que representa con que
% nivel de seguridad la entidad corresponde a una entidad real.
% En definitiva una entidad puede ser identificada por un par $(a, b)$ y un entero
% $c$, donde $(a, b)$ y representa el intervalo del texto donde se encuentra la entidad
% y $c$ la confianza con que la entidad fue reconocida.

El equipo de la OCI está experimentando algunos problemas con el reconocimiento de
entidades nombradas en ocimatic y necesitan de tu ayuda.
Específicamente, ocimatic está reconociendo entidades que están \emph{contenidas}
dentro de otras.
Por ejemplo, para el texto de ejemplo anterior, ocimatic actualmente reconoce las
siguientes entidades: $(1, 6)$, $(8, 16)$, $(1, 16)$ y $(46, 48)$.
Es claro que la entidad $(1, 6)$, correspondiente al texto \emph{Gimuel}, y la
entidad $(8, 16)$, correspondiente al texto \emph{Caspumano}, están
contenidas dentro de la entidad $(1, 16)$, correspondiente al texto \emph{Gimuel
  Caspumano}.
Decimos que una entidad $(a_1,b_1)$ está contenida dentro de una entidad
$(a_2,b_2)$ si $a_2\leq a_1$ y $b_1\leq b_2$.

La OCI está interesada en saber cuál es el impacto de este problema.
% el cuál
% puede contabilizarse como la cantidad de entidades que están contenidas dentro
% de otra.
Dado un conjunto de entidades, tu tarea es evaluar este impacto contando la
cantidad de entidades que están contenidas dentro de otra.
% Específicamente, el impacto para un texto corresponde a la suma de las
% confianzas de las entidades contenidas dentro de otra entidad.
\end{problemDescription}

\begin{inputDescription}
  La primera línea contiene un entero $N$ correspondiente al número de entidades.
  Cada una de las siguientes $N$ líneas contiene dos enteros $a$ y $b$ ($1\leq
  a\leq b \leq 1000$) que describen una entidad $(a,b)$.
\end{inputDescription}

\begin{outputDescription}
  La salida debe contener una única línea con un entero correspondiente a la cantidad de
  entidades que están contenidas dentro de otra.
\end{outputDescription}

\begin{scoreDescription}
  \score{20} Se probarán varios casos donde $1\leq N\leq 2$.
  \score{30} Se probarán varios casos donde $2< N \leq 100$.
  \score{30} Se probarán varios casos donde $100< N \leq 10^5$.
\end{scoreDescription}

\begin{sampleDescription}
\sampleIO{sample1}
\sampleIO{sample2}
\end{sampleDescription}

\end{document}
